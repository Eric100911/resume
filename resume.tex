% !TEX program = xelatex

\documentclass{resume}
%\usepackage{zh_CN-Adobefonts_external} % Simplified Chinese Support using external fonts (./fonts/zh_CN-Adobe/)
%\usepackage{zh_CN-Adobefonts_internal} % Simplified Chinese Support using system fonts

\begin{document}
\pagenumbering{gobble} % suppress displaying page number

\name{Chi Wang}

\basicInfo{
  \email{chi.w@cern.ch} \textperiodcentered\ 
  \github[Eric100911]{github.com/Eric100911/}
}

% Education:

\section{\faGraduationCap\ Education}
\datedsubsection{\textbf{Tsinghua University}, Beijing, China}{2022 -- Present}
\datedline{\textit{Bachelor Student} in Physics, expected July 2026}{GPA:3.75/4.00 (Major)}

% Honors and Awards:

\section{\faTrophy\ Honors and Awards}
\datedsubsection{\textbf{Comprehensive Excellence Scholarship}, Tsinghua University}{Nov. 2024}
\begin{itemize}
    \item Mitsubishi UFJ Foundation Scholarship Program 2024
\end{itemize}
\datedsubsection{\textbf{Honorable Mention}, Interdisciplinary Contest in Modeling (ICM), COMAP}{Apr. 2024}
\datedsubsection{\textbf{\nth{1} Prize in Zhejiang Province}, Chinese Physics Olympiad (CPhO)}{Sept. 2021}

% Research experience:
% Co-leader of the project "Search for Simultaneous Production of double Jpsi and Upsilon mesons in Proton-Proton Collisions at 13.6 TeV".
% - Developed the analysis framework for the project.

\section{\faBook\ Research Experience}
\datedsubsection{\textbf{Search for Simultaneous Production of Double $J/\psi$ and $\Upsilon$ Mesons,\\Double $J/\psi$ and $\phi$ Mesons, or $J/\psi$, $\Upsilon$ and $\phi$ Mesons in Proton-Proton\\ Collisions at $\sqrt{s} = 13.6 ~ \mathrm{TeV}$ at the CMS Experiment}}{Jul. 2024 -- Present}
\role{Co-leader}{Supervisor: Dr. Zhen Hu}
\begin {itemize}
  \item Co-led the analysis and developed the analysis framework (work in progress).
  \item Undergraduate "Inspiring" Program funding (\$5000) from Beijing Natural Science Foundation (BNSF) for the project.
  \item Present talk at CMS BPH production and properties meeting (9 Jul. 2025).
  \item Research visit at CERN (01/2025 - 02/2025).
\end{itemize}

\datedsubsection{\textbf{Implementation of IRC-safe Jet Flavour Algorithms in CMSSW for NanoAOD Production}}{Jul. 2025 -- Present}
\role{Co-developer}{Supervisor: Dr. Markus Seidel}
\begin {itemize}
  \item Research visit at CERN (07/2025 - 08/2025).
\end{itemize}

% Worked as peer teaching assistant in the course of "Big Data in Experimental Physics (1) & (2)" in the summer semester of 2024.

\section{\faUsers\ Teaching and Mentoring Experience}
\datedsubsection{\textbf{Big Data in Experimental Physics (1) \& (2)}, Tsinghua University}{Summer 2024}
\role{Peer Teaching Assistant}{Supervisor: Prof. Benda Xu}
\begin {itemize}
  \item Contributed to the revision of the course material.
  \item Provided assistance to students in the courses.
  \item Achieved grades of A- (4.0 / 4.0) in both courses.
\end{itemize}
\datedsubsection{\textbf{Data Structure}, Tsinghua University}{Fall 2023}
\role{Study Group Leader}{Supervisor: Prof. Junhui Deng}
\begin {itemize}
  \item Organized study group sessions for group members.
  \item Provided assistance to group members.
\end{itemize}

% Publications:

\section{\faFileText\ Publications}
\begin{itemize}
  \item Liu, J., Pang, H., \textbf{Wang, C.}, Ai, X., Chen, X., \& Hu, Z. (2024) FASER experiment: An introduction and research progress. \textit{Chinese Science Bulletin, 69}(8), 1025-1033.
\end{itemize}

% Skills:

\section{\faCogs\ Skills}
\begin{itemize}[parsep=0.5ex]
  \item Programming Languages: Proficient in C++ and Python. Familiar with R.
  \item Research Software: Experienced with ROOT. Familiar with HELAC-Onia.
  \item Languages: English (proficient, CET4 675/710).
\end{itemize}


%% Reference
%\newpage
%\bibliographystyle{IEEETran}
%\bibliography{mycite}
\end{document}
